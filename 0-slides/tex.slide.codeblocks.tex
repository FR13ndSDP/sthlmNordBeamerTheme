\documentclass[\string~/GitHub/sthlmNordBeamerTheme/sthlmNordLightDemo.tex]{subfiles}

\begin{document}
%=-=-=-=-=-=-=-=-=-=-=-=-=-=-=-=-=-=-=-=-=-=-=-=-=-=-=-=-=-=-=-=-=-=-=-=-=-=-=-=
%   FRAME START   -=-=-=-=-=-=-=-=-=-=-=-=-=-=-=-=-=-=-=-=-=-=-=-=-=-=-=-=-=-=-=
\begin{frame}[fragile, allowframebreaks]{Using Minted for Code Documentation}

	The use of minted code highlighting can be activated by passing the 
	codehl option to the documentclass
	\begin{beamercodeblock}
\begin{minted}{tex}
\documentclass[codehl]{beamer}
\end{minted}
\end{beamercodeblock}
	\begin{alertblock}{Warning}
		The Nord color style used by pygments needs to be installed from:
		\url{https://github.com/sbrisard/nord_pygments}
	\end{alertblock}
\end{frame}

%   FRAME END   --==-=-=-=-=-=-=-=-=-=-=-=-=-=-=-=-=-=-=-=-=-=-=-=-=-=-=-=-=-=-=
%=-=-=-=-=-=-=-=-=-=-=-=-=-=-=-=-=-=-=-=-=-=-=-=-=-=-=-=-=-=-=-=-=-=-=-=-=-=-=-=

%=-=-=-=-=-=-=-=-=-=-=-=-=-=-=-=-=-=-=-=-=-=-=-=-=-=-=-=-=-=-=-=-=-=-=-=-=-=-=-=
%   FRAME START   -=-=-=-=-=-=-=-=-=-=-=-=-=-=-=-=-=-=-=-=-=-=-=-=-=-=-=-=-=-=-=
\begin{frame}[fragile, allowframebreaks]{Using Minted for Code Documentation}

	It might be nice to include some inline code \mintinline{latex}|\documentclass[opt]{name}|

	\begin{beamercodeblock}
\begin{minted}{tex}
\documentclass[opt]{name}
\prob Solve the equation \( \cos x = \frac{1}{2} \) 
for \( 0 \le x \le 2\pi \).

\soln A fantastic solution will follow.
\end{minted}
	\end{beamercodeblock}
	\beamercaptionblock{\textbf{Listing 1:} \LaTeX\ example}
	\framebreak
	\begin{beamercodeblock}
\begin{minted}{python}
import os
import sys
import subprocess
import getpass
from pathlib import Path
import shortuuid
from datetime import datetime
from tabulate import tabulate
\end{minted}
	\end{beamercodeblock}
	\beamercaptionblock{\textbf{Listing 2:} Python example}
\end{frame}
%   FRAME END   --==-=-=-=-=-=-=-=-=-=-=-=-=-=-=-=-=-=-=-=-=-=-=-=-=-=-=-=-=-=-=
%=-=-=-=-=-=-=-=-=-=-=-=-=-=-=-=-=-=-=-=-=-=-=-=-=-=-=-=-=-=-=-=-=-=-=-=-=-=-=-=
\end{document}